\thispagestyle{empty}
\vspace*{0.2cm}

\begin{center}
    \textbf{Zusammenfassung}
\end{center}
Moderne Cloud-Systeme, die immer mehr Rechenleistung und Services anbieten, werden immer unüberschaubarer und komplexer und dadurch schwer zu warten. Die automatische und präzise Erkennung von Anomalien ist wichtig, um die Zuverlässigkeit, Sicherheit und den korrekten Betrieb eines Cloud Systems zu gewährleisten. Systemlogs sind die primäre Quelle bei der Fehlerdiagnose. Zahlreiche Studien verfolgen unflexible Lösungsansätze, die auf unzureichenden Darstellungen von Logs basieren. Dies führt zu einer ungenügenden Generalisierung, die jedoch aufgrund sich ständig verändernder Logs durch Updates und Umgebungsveränderungen nötig ist. Die vorliegende Arbeiten stellt ein System vor, welches Sprachmodelle verwendet um die natürlichsprachlichen Eigenschaften von Systemlogs zu erfassen und numerische Repräsentationen von ebendiesen als Eingabe eines Bi-LSTMs für Anomalieerkennung einzusetzen. Mehrere Experimente werden mit Cloud-Systemlogs durchgeführt, um die Qualität von den Sprachmodellen im Hinblick auf Anomalieerkennung zu evaluieren. Insbesondere wird die Robustheit gegenüber Veränderungen in normalen Logverläufen überprüft und die Übertragbarkeit der gelernten Eigenschaften eines Datensatzes auf einen anderen Datensatz simuliert. Es wird gezeigt, dass das Modell mit Hilfe der durch Sprachmodelle erzeugten numerischen Repräsentationen in der Lage ist, gute Ergebnisse in der Anomalieerkennung zu erzielen.
\vspace*{0.2cm}

\noindent 
