\thispagestyle{empty}
\vspace*{0.2cm}

\begin{center}
    \textbf{Zusammenfassung}
\end{center}
Moderne Cloudsysteme, die immer mehr Rechenleistung und Services anbieten, werden immer größer und komplexer und sind dadurch schwerer zu warten. Die automatische und präzise Erkennung von Anomalien ist wichtig, um die Zuverlässigkeit, Sicherheit und den korrekten Betrieb eines Cloudsystems zu gewährleisten. Systemlogs sind die primäre Quelle bei der Fehlerdiagnose. Zahlreiche Studien verfolgen Lösungsansätze, die auf einer unflexiblen Darstellung von Logs basieren. Dies führt zu unzureichender Generalisierung, die jedoch aufgrund sich ständig verändernder Logs durch Updates und Umgebungsveränderungen nötig ist. Die vorliegende Arbeit stellt ein System vor, welches Sprachmodelle verwendet, um die natürlichsprachlichen Eigenschaften von Systemlogs zu erfassen und numerische Repräsentationen von ebendiesen als Eingabe eines Bi-LSTMs für Anomalieerkennung einzusetzen. Mehrere Experimente werden mit Cloudsystemlogs durchgeführt, um die Qualität der Sprachmodelle im Hinblick auf Anomalieerkennung zu evaluieren. Insbesondere wird die Robustheit gegenüber Veränderungen in normalen Logverläufen überprüft und die Übertragbarkeit der gelernten Eigenschaften eines Log-Datensatzes auf einen anderen Log-Datensatz simuliert. Es wird gezeigt, dass das Modell mit Hilfe der durch Sprachmodelle erzeugten numerischen Repräsentationen in der Lage ist, sehr gute Ergebnisse in der Anomalieerkennung zu erzielen.
\vspace*{0.2cm}

\noindent 
